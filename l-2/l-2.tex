\documentclass[12pt,a4paper]{article}
\usepackage[utf8]{inputenc}
\usepackage[ukrainian]{babel}
\usepackage{amsmath}
\usepackage{setspace}  
\onehalfspacing

 \newtheorem{law}{Означення}\newtheorem{law1}[law]{Означення}\newtheorem{law2}[law]{Означення}\newtheorem{law3}[law]{Означення}
\newtheorem{theorem}{Теорема}\newtheorem{theorem1}[theorem]{Теорема}\newtheorem{theorem2}[theorem]{Теорема}\newtheorem{theorem3}[theorem]{Теорема}\newtheorem{theorem4}[theorem]{Теорема}
\newtheorem{example}{Приклад}\newtheorem{example1}[example]{Приклад}\newtheorem{example2}[example]{Приклад}
\begin{document}
\begin{center}
\section*{Вступ}
\end{center}
\begin{flushleft}
\parbox{14,5 cm}{\parindent=1 cm  Темпи розвитку економіки, розв’язання багатьох соціальних проблем залежать від інтенсивності впровадження досягнень науково-технічного прогресу в галузях народного господарства. В свою чергу, цю проблему неможливо розв’язати без швидкого розвитку і впровадження в усі сфери людської діяльності сучасних засобів обчислювальної техніки і прикладної математики.}
\end{flushleft}
\begin{flushleft}
\parbox{14,5 cm}{\parindent=1 cm  Одним з розділів прикладної математики, до якого інженерно-технічні працівники і економісти проявляюсь підвищений інтерес, це мінімізація функцій і функціоналів. Велика кількість різноманітних задач і методів їх розв’язання обумовлює мати посібники, в яких в стислій формі було б викладено алгоритми самих відомих методів і методику їх застосування. До цього спонукають також нові форми навчання студентів.}
\end{flushleft}
\begin{flushleft}
\parbox{14,5 cm}{\parindent=1 cm  Основне завдання посібника – це допомога студентам в опануванні основних алгоритмів розв’язування задач мінімізації функцій багатьох змінних, задач варіаційного числення і оптимального керування.}
\end{flushleft}
\newpage
\begin{center}
\part{Мінімізація функцій}
\section{Мінімізація функцій однієї змінної}
\end{center}
\begin{flushleft}
\parbox{14,5 cm}{\parindent=1 cm  Нехай на числовій прямій ${E}^1$ задана скалярна функція $\varphi\left(x\right)$. Розглянемо задачу пошуку точок, в яких функція досягає свого мінімального або максимального значення. Точка мінімуму або максимуму функції будемо називати максимальними точками.}
\end{flushleft}
\begin{flushleft}
\parbox{14,5 cm}{\parindent=1 cm Сформулюємо означення, що відносяться до теорії мінімізації функції.}
\end{flushleft}
\begin{law}
Якщо для всіх $x\in{E}^1$ виконується умова $\varphi\left({x}^*\right)\le\varphi\left(x\right)$, то точка ${x}^*$ називається точкою глобального (абсолютного) мінімуму функції $\varphi\left(x\right)$.
\end{law}
\begin{law1}
Якщо для достатньо малого $\varepsilon > 0$ виконується нерівність $\varphi\left({x}^*\right)\le\varphi\left(x\right)$ для всіх $x\in {E}^1$ таких, що $\left|x-{x}^*\right|\le\varepsilon$, то точка ${x}^*$ називається точкою локального (відносного) мінімуму функції $\varphi\left(x\right)$.
\end{law1}
\begin{law2}
Точка ${x}^*$ називається точкою строгого мінімуму (в локальному або глобальному сенсі), якщо відповідні нерівності в означеннях точок локального і глобального мінімумів виконуються як строгі (при $x \neq {x}^*$). 
\end{law2}
\begin{flushleft}
\parbox{14,5 cm}{\parindent=1 cm Аналогічним чином вводиться означення точок локального і глобального максимумів.}
\end{flushleft}
\begin{flushleft}
\parbox{14,5 cm}{\parindent=1 cm  Зазначимо, що точки глобального мінімуму є точками локального мінімуму. І тому далі розглядатимемо тільки точки локального мінімуму.}
\end{flushleft}
\begin{theorem}[необхідна умова екстремуму першого порядку]
Нехай функція $\varphi\left(x\right)$ визначена і диференційована на ${E}^1$. Якщо ${x}^*$ - точка локального мінімуму (максимуму), то в ній перша похідна функції дорівнює нулю: 
\begin{equation}\label{one}
\frac{d\varphi\left({x}^*\right)}{dx}=0.
\end{equation}
\end{theorem}
\begin{law3}
Точки, що задовольняють умові (\ref{one}) називаються стаціонарними.
\end{law3}
\begin{example}
Знайти стаціонарні точки функції $\varphi\left(x\right) = \frac{1}{3}{x}^3 - 5{x}^2+24x + 6$.
Знайдемо нулі першої похідної: $\frac{d\varphi\left(x\right)}{dx} = {x}^2 - 10x+24$. Маємо ${x}^1 = 4, {x}^2 = 6$. Отже, в точках ${x}^1$ і ${x}^2$ функція може досягати екстремальних значень.
\end{example}
\begin{theorem1}[необхідна умова екстремуму другого порядку]
Нехай функція $\varphi\left(x\right)$ визначена і двічі диференційована на ${E}^1$. Тоді в точці локального мінімуму (максимуму) друга похідна функції невід’ємна (недодатна): $\frac{{d}^2\varphi\left({x}^*\right)}{d{x}^2} \ge 0 \ \left(\le 0\right)$.
\end{theorem1}
\begin{example1}
Розглянемо ту ж саму функцію $\varphi\left(x\right) = \frac{1}{3}{x}^3-5{x}^2+24x+6$. Друга похідна має вигляд $\frac{{d}^2\varphi\left(x\right)}{d{x}^2} = 2x-10$. У стаціонарній точці ${x}^1 = 4$ друга похідна дорівнює $-2$, тобто від’ємна. У цій точці може досягатися максимум функції. У точці ${x}^2 = 6$ друга похідна додатна, тобто в ній може досягатися мінімум функції.
\end{example1}
\begin{theorem2}[достатня умова екстремуму]
Нехай функція $\varphi\left(x\right)$ визначена, двічі диференційована на ${E}^1$. Якщо у стаціонарній точці ${x}^*$ виконується умова $\frac{{d}^2\varphi\left({x}^*\right)}{d{x}^2} > 0 \ \left(< 0\right)$, то точка ${x}^*$ - точка локального мінімуму (максимуму) функції $\varphi\left(x\right)$. 
\end{theorem2}
\begin{example2}
Розглядається відома нам функція $\varphi\left(x\right) = \frac{1}{3}{x}^3-5{x}^2+24x+6$. У стаціонарній точці ${x}^1 = 4$ друга похідна дорівнює $-2$. Отже, в цій точці досягається максимум функції, а в точці ${x}^2 = 6$ друга похідна дорівнює $2$, тобто в ній досягається мінімум.
\end{example2}
\begin{flushleft}
\parbox{14,5 cm}{\parindent=1 cm Якщо в стаціонарній точці ${x}^*$ друга похідна дорівнює нулю, то питання про мінімум чи максимум у цій точці залишається відкритим.}
\end{flushleft}
\begin{theorem3}[загальна достатня умова екстремуму]
Нехай функція $\varphi\left(x\right)$ визначена на ${E}^1$ і має неперервні похідні до k-го порядку включно. Якщо в точці ${x}^*$ похідні до (k-1)-го порядку дорівнюють нулю:
$$\frac{d\varphi\left({x}^*\right)}{dx} = 0, \dots, \frac{{d}^{k-1}\varphi\left({x}^*\right)}{d{x}^{k-1}} = 0, але \frac{{d}^k\varphi\left({x}^*\right)}{d{x}^k}\neq 0\text{, то:}$$
\begin{enumerate}
\item ${x}^*$ є точкою локального мінімуму, якщо k-парне число і $\frac{{d}^k\varphi\left({x}^*\right)}{d{x}^k}>0$;
\item ${x}^*$ є точкою локального максимуму, якщо k-парне число і $\frac{{d}^k\varphi\left({x}^*\right)}{d{x}^k}<0$;
\item ${x}^*$ не є ні точкою мінімуму, ні точкою максимуму, якщо  k – непарне число.
\end{enumerate}
\end{theorem3}
\begin{flushleft}
\parbox{14,5 cm}{\parindent=1 cm Далі сформулюємо необхідну умову мінімуму функції на відрізку $\left[a;b\right]$.}
\end{flushleft}
\begin{theorem4}
Якщо точка ${x}^*= a$ є точкою мінімуму функції $\varphi\left(x\right)$ на відрізку $\left[a;b\right]$, то $\frac{d\varphi\left({x}^*\right)}{dx}\ge 0$, а якщо ${x}^* = b$ – точка мінімуму, то $\frac{d\varphi\left({x}^*\right)}{dx}\le 0$.
\end{theorem4}
\end{document}